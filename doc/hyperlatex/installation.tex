\section{Installation}

\htmlmenu{2}

The GraspIt! installation process consists of two steps. First, you
must install a set of external libraries, listed below. The
installation for each library is further detailed in the system
specific sections of this installation guide. Then, you must compile
the GraspIt! code itself and set up the environment variables.

GraspIt! needs the following external libraries:

\begin{description}
\item[Qt] Qt is a cross-platform application and UI framework. It
  allows GraspIt! to have a platform-independent windowing system,
  dialogs etc. GraspIt! currently uses Qt version 4, which is
  available under two different licenses. This installer assumes you
  will be using the open source version, see the \xlink{Qt
    website}{http://www.qtsoftware.com} for more details on licensing
  options.

\item[Coin] Coin is a Scene Graph API that GraspIt! uses for all of
  its 3D rendering needs. It is an Open Inventor clone. Like Qt, Coin
  is available under a dual licensing model; here we assume you will
  be using the open source license. See the
  \xlink{Coin3D}{http://www.coin3d.org} website for more licensing
  details.

\item[SoQt] SoQt is a GUI binding, essentially the "glue" between Qt
  and Coin. It is available, under the same conditions as Coin, from
  the Coin website. The latest version of SoQt that we have tested
  GraspIt! with is SoQt 1.4.2, which provides seamless integration
  with Qt4.

\item[Lapack] Lapack is a library for scientific computing that
  GraspIt! uses for various matrix related tasks, such as matrix
  multiplication, linear system solving, singular value decomposition,
  etc.
\end{description}

The GraspIt! project information comes in the form of a cross-platform
Qt project file, which is processed by QMake. Depending on your
system, you will convert this into either a Makefile or a Microsoft
Visual Studio project file.
