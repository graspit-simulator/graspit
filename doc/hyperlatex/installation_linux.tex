\subsection{GraspIt! Installation - Ubuntu Linux}

\subsubsection{Qt}

You will need the following libraries in order to compile GraspIt!:
\begin{itemize}
\item Qt 4
\item Coin 3
\item SoQt
\item Blas and Lapack
\end{itemize}

On Ubuntu, you should be able to get all these from the package
manager. It should also possible to compile them all from code, but
since that tends to be system-specific we no longer provide
instructions or guidelines for that.

Use \texttt{sudo apt-get install} to install the following packages:
\begin{itemize}
\item \texttt{libqt4-dev}
\item \texttt{libqt4-opengl-dev}
\item \texttt{libqt4-sql-psql}
\item \texttt{libcoin60-dev}
\item \texttt{libsoqt4-dev}
\item \texttt{libblas-dev}
\item \texttt{liblapack-dev}
\end{itemize}

\subsubsection{GraspIt!}

Download and unzip the GraspIt! code itself. Set the following
environment variable:

\begin{itemize}
\item \texttt{GRASPIT} - the directory where you unzipped GraspIt!
\end{itemize}

Build QHull. Go to the \texttt{\$GRASPIT/qhull} directory and type
\texttt{make}.

Create the GraspIt! Makefile from the Qt project file. Edit
\texttt{\$GRASPIT/graspit.pro} to suit your system and
installation needs. Then type \texttt{qmake graspit.pro}.

Build and go. From \texttt{\$GRASPIT/} type \texttt{make}.
